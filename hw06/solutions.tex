\documentclass[12pt]{article}
\usepackage[margin=1in]{geometry}

%%% Packages
% Math
\usepackage{amsmath,amsfonts,amssymb,amsthm}
\numberwithin{equation}{section}
\theoremstyle{plain}
\newtheorem{claim}{Claim}
\newtheorem{corollary}{Corollary}
\newtheorem{booktheoremX}{(Textbook) Theorem}
\newenvironment{booktheorem}[1]{%
\renewcommand\thebooktheoremX{#1}%
\booktheoremX
}{\endbooktheoremX}

% Pseudo code
\usepackage{algorithm, algpseudocode}

% I think this needs to be the last package
\usepackage{hyperref}
%%%

%%% Commands
%% General
\newcommand{\set}[1]{\{ #1 \}}
\mathchardef\mhyphen="2D 

%% Probability commands
\DeclareMathOperator*{\prob}{Pr}
\newcommand{\given}{\mid}
\DeclareMathOperator*{\lcm}{lcm}

%% Crypto commands
\newcommand{\ppt}{\algo{PPT}}
\newcommand{\func}{\algo{Func}}
\newcommand{\ctexts}{\mathcal{C}}
\newcommand{\ctext}{\algo{C}}
\newcommand{\ptexts}{\mathcal{M}}
\newcommand{\ptext}{\algo{M}}
\newcommand{\keys}{\mathcal{K}}
\newcommand{\key}{\algo{K}}
\newcommand{\algo}[1]{\mathsf{#1}}
\newcommand{\adv}{\mathcal{A}}
\newcommand{\advv}{\mathcal{A}'}
\DeclareMathOperator{\negl}{\algo{negl}}
% Private key
\newcommand{\priv}{\Pi}
\newcommand{\gen}{\algo{Gen}}
\newcommand{\enc}{\algo{Enc}}
\newcommand{\dec}{\algo{Dec}}
\newcommand{\mac}{\algo{Mac}}
\newcommand{\param}{1}
\newcommand{\privexpir}[3]{\algo{PrivK}^{#1}_{{#2},{#3}}}
\newcommand{\macexpir}[2]{\algo{MACForge}_{{#1},{#2}}}
\newcommand{\eav}{\algo{eav}}
\newcommand{\cpa}{\algo{cpa}}
%%%

\title{%
CSE 5351 Spring 2020\\
Homework 6
}
\author{%
Caleb Lehman
(\href{mailto:lehman.346@osu.edu}{\texttt{lehman.346@osu.edu}})
}
\date{%
}

\begin{document}

\maketitle

\textbf{Note:} Throughout these solutions, when we write $\bmod n$ at the end of a string of equalities,
it applies to each equality in the string.

\section*{Problem 1}

\begin{align*}
    \phi(2400)
        &= \phi(2^5 \cdot 3 \cdot 5^2) \\
        &= \phi(2^5) \cdot \phi(3) \cdot \phi(5^2) \\
        &= 2^{5-1} \cdot (2 - 1) \cdot (3 - 1) \cdot 5^{2-1} \cdot (5 - 1) \\
        &= 16 \cdot 2 \cdot 5 \cdot 4 \\
        &= 640
\end{align*}

\section*{Problem 2}

First note that $\phi(21) = \phi(3 \cdot 7) = \phi(3) \cdot \phi(7) = 2 \cdot 6 = 12$
and $123622448602 = 12 \cdot 1030204050 + 2 = 2 \mod 12$.
Then by Euler's theorem, since $\gcd(19, 21) = 1$, we have
\begin{align*}
    19^{12362448602}
        &= 19^{12362448602 \bmod 12} \\
        &= 19^2 \\
        &= (-2)^2 \\
        &= 4 \bmod 21
\end{align*}

\newpage
\section*{Problem 3}

We have
\begin{gather*}
    2^1 = 2 \mod 3\\
    2^2 = 1 \mod 3
\end{gather*}
so the order of $2$ in $\mathbb{Z}^*_{3}$ is 2.
We have
\begin{gather*}
    2^1 = 2 \mod 7\\
    2^2 = 4 \mod 7\\
    2^3 = 1 \mod 7
\end{gather*}
so the order of $2$ in $\mathbb{Z}^*_{7}$ is 3.
$21 = 3 \cdot 7$ is the prime factorization of $21$,
so the order of $2$ in $\mathbb{Z}^*_{21}$ is $\lcm(2, 3) = 6$.
Indeed, we can verify this explicitly as follows:
\begin{gather*}
    2^1 = 2 \mod 21\\
    2^2 = 4 \mod 21\\
    2^3 = 8 \mod 21\\
    2^4 = 16 \mod 21\\
    2^5 = 11 \mod 21\\
    2^6 = 1 \mod 21
\end{gather*}

\section*{Problem 4}

We note that $3599 = 3600 - 1 = 60^2 - 1 = (60+1)(60-1) = 61 \cdot 59$.
Since $61$ and $59$ are prime,
we have that $\phi(3599) = \phi(61) \cdot \phi(59) = 60 \cdot 58 = 3480$
and we want to compute $d$ such that $31 \cdot d = 1 \bmod 3480$.
We can compute B\'ezout coefficients, starting with the guess that $31 \cdot 2 \cdot 58 \approx 60 \cdot 58$:
\begin{align*}
    31 \cdot 2 \cdot 58 - 60 \cdot 58
        &= 1 \cdot 2 \cdot 58\\
        &= 116\\
    31 \cdot 4 - 116
        &= 8\\
    8 \cdot 4 - 31
        &= 1
\end{align*}
Substituting each equations where appropriate, we have
\begin{align*}
    1
        &= 8 \cdot 4 - 31\\
        &= (31 \cdot 4 - 116) \cdot 4 - 31\\
        &= (31 \cdot 4 - (31 \cdot 2 \cdot 58 - 3480)) \cdot 4 - 31\\
        &= 31 \cdot ((4 - 116) \cdot 4 - 1) + 3480 \cdot 4\\
        &= 31 \cdot (-449) + 3480 \cdot 4
\end{align*}

It follows that $31 \cdot 3031 = 31 \cdot (-449) = 1 - 3480 \cdot 4 = 1 \bmod 3480$.
That is, the private key of the user is $d=3031$, $N=3599$.


\section*{Problem 5}

We note that $155 = 5 \cdot 31$.
Since $5$ and $31$ are prime,
we have that $\phi(155) = \phi(5) \cdot \phi(31) = 4 \cdot 30 = 120$,
and we want to compute $d$ such that $7 \cdot d = 1 \bmod 120$.
Note that
\begin{gather*}
    7 \cdot 17 = 119 = -1 \bmod 120,
\end{gather*}
so $7 \cdot (7 \cdot 17^2) = (7 \cdot 17)^2 = (-1)^2 = 1 \bmod 120$.
We have
\begin{align*}
    7 \cdot 17 \cdot 17
        &= 7 \cdot 289\\
        &= 7 \cdot 49\\
        &= 343\\
        &= 103 \mod 120
\end{align*}
Hence, the private key is $d = 103$, so we need only compute $c^{103} = 61^{103} \mod 155$.
Note that $61 \cdot 61 = 60^2 + 2 \cdot 60 + 1 = 30 \cdot 120 + 120 + 1 = 1 \bmod 120$.
It follows that $61^{103} = 61^{103 \bmod 2} = 61 \bmod 120$.
That is, the original plaintext is $m = 61$.

\section*{Problem 6}

Fix $N, e$ and suppose we have an adversary $\adv$ running in time $t$ that takes ciphertext as input and satisfies
\begin{gather*}
    \prob[\adv(x^e) = x \bmod N : x \gets_u \mathbb{Z}^*_N] = 0.01
\end{gather*}
Define the following adversary $\advv$ by the following procedure:
\begin{algorithm}[H]
\begin{algorithmic}[1]
\Procedure{$\advv$}{c}
    \State $i \gets 1$
    \While{$i \leq 500$}
        \State $r \gets_u \mathbb{Z}^*_N$\label{alg:sample}
        \State $c' \gets c \cdot r^e \bmod N$\label{alg:uniform}
        \State $x' \gets \adv(c')$\label{alg:subroutine}
        \If{$(x')^e = c' \bmod N$}\label{alg:verify}
            \State \Return $x' \cdot r^{-1} \bmod N$\label{alg:inverse}
        \EndIf
        \State $i \gets i+1$\label{alg:update}
    \EndWhile
    \State \Return $\bot$
\EndProcedure
\end{algorithmic}
\end{algorithm}

We first comment on the runtime of $\advv$.
Note that the multiplication on line (\ref{alg:uniform}),
the exponentiation on line (\ref{alg:verify}),
and the inverse computation on line (\ref{alg:inverse})
each run in $\mathcal{O}(\|N\|^3)$.
Line (\ref{alg:subroutine}) runs in time $t$,
so each iteration of the loop has runtime $\mathcal{O}(t + 3 \cdot \|N\|^3)$
and the whole procedure has runtime $\mathcal{O}(100 (t + 3 \cdot \|N\|^3)) = \mathcal{O}(t + \|N\|^3)$.
In particular, the runtime of $\advv$ is polynomial in $t$ and $\| N \|$.

We now compute the probability of success of $\advv$.
First, let $c = x^e$ and consider a single iteration of the while loop (lines (\ref{alg:sample})-(\ref{alg:update})).
Then $c' = c \cdot r^e = x^e \cdot r^e = (x \cdot r)^e \bmod N$.
Since multiplication by $x$ is a bijection on $\mathbb{Z}^*_N$ (with inverse given by multiplication by $x^{-1}$),
the samples $(x \cdot r)_{r \gets_u \mathbb{Z}^*_N}$ are uniformly distributed in $\mathbb{Z}^*_N$.
In particular,
\begin{align*}
    \prob[\adv(c') = x \cdot r \bmod N : r \gets_u \mathbb{Z}^*_N]
        &= \prob[\adv((x \cdot r)^e) = x \cdot r \bmod N : r \gets_u \mathbb{Z}^*_N]\\
        &= \prob[\adv(y^e) = y \bmod N : y \gets_u \mathbb{Z}^*_N]\\
        &= 0.01
\end{align*}
Furthermore, if $x' \gets \adv(c') = x \cdot r$, then $(x')^e = (x \cdot r)^e = c'$ is true
and $x' \cdot r^{-1} = x \cdot r \cdot r^{-1} = x \bmod N$, so the whole procedure returns the correct plaintext.
In other words, $\prob[\text{iteration }i\text{ succeeds}] = 0.01$.
Each loop is independent, so we have
\begin{align*}
    \prob[\advv(x^e) = x : x \gets_u \mathbb{Z}^*_N]
        &= 1 - \prob[\text{iteration }i\text{ fails for all }i]\\
        &= 1 - \Pi_{i = 1}^{500} \prob[\text{iteration }i\text{ fails}]\\
        &= 1 - \Pi_{i = 1}^{500} (1 - \prob[\text{iteration }i\text{ succeeds}])\\
        &= 1 - \Pi_{i = 1}^{500} 0.99\\
        &= 1 - 0.99^{500}\\
        &\geq 0.99
\end{align*}

To summarize, given ciphertext $c = x^e \bmod N$, the adversary $\advv(c)$ given by the procedure above
has runtime polynomial in $t$ and $\|N\|$
and produces $x$ with probability $\geq 0.99$.

\end{document}
