\documentclass[12pt]{article}
\usepackage[margin=1in]{geometry}

%%% Packages
% Math
\usepackage{amsmath,amsfonts,amssymb,amsthm}
\numberwithin{equation}{section}
\theoremstyle{plain}
\newtheorem{claim}{Claim}
\newtheorem{corollary}{Corollary}
\newtheorem{booktheoremX}{(Textbook) Theorem}
\newenvironment{booktheorem}[1]{%
\renewcommand\thebooktheoremX{#1}%
\booktheoremX
}{\endbooktheoremX}

% Pseudo code
%\usepackage{algorithm, algpseudocode}

% I think this needs to be the last package
\usepackage{hyperref}
%%%

%%% Commands
%% General
\newcommand{\set}[1]{\{ #1 \}}

%% Probability commands
\DeclareMathOperator*{\prob}{Pr}
\newcommand{\given}{\mid}

%% Crypto commands
\newcommand{\func}{\algo{Func}}
\newcommand{\ctexts}{\mathcal{C}}
\newcommand{\ctext}{\algo{C}}
\newcommand{\ptexts}{\mathcal{M}}
\newcommand{\ptext}{\algo{M}}
\newcommand{\keys}{\mathcal{K}}
\newcommand{\key}{\algo{K}}
\newcommand{\algo}[1]{\mathsf{#1}}
\newcommand{\adv}{\mathcal{A}}
\newcommand{\advv}{\mathcal{A}'}
\DeclareMathOperator{\negl}{negl}
% Private key
\newcommand{\priv}{\Pi}
\newcommand{\gen}{\algo{Gen}}
\newcommand{\enc}{\algo{Enc}}
\newcommand{\dec}{\algo{Dec}}
\newcommand{\param}{1}
\newcommand{\expir}[3]{\algo{PrivK}^{#1}_{{#2},{#3}}}
\newcommand{\eav}{\algo{eav}}
\newcommand{\cpa}{\algo{cpa}}
%%%

\title{%
CSE 5351 Spring 2020\\
Homework 3
}
\author{%
Caleb Lehman
(\href{mailto:lehman.346@osu.edu}{\texttt{lehman.346@osu.edu}})
}
\date{%
}

\begin{document}

\maketitle

\section*{Problem 1}

Let $G$ be any pseudorandom generator with expansion factor $l(n) = 2n$.
\begin{claim}
$F: \set{ 0, 1 }^n \times \set{ 0, 1 }^{2n} \to \set{ 0, 1 }^{2n} : (k, x) \mapsto G(k) \oplus x$ is not a pseudorandom function.
\end{claim}
\begin{proof}
Let $\func_{2n}$ denote the set of all functions $f : \set{ 0, 1 }^{2n} \to \set{ 0, 1 }^{2n}$.
Then
\begin{align*}
    \prob[D^{f(\cdot)}(1^n) = 1 : f \gets_u \func_{2n}]
        &= \prob[f(0^{2n}) \oplus f(1^{2n}) = 1^{2n} : f \gets_u \func_{2n}]\\
        &= \prob[f(0^{2n}) = 1^{2n} \oplus f(1^{2n}) : f \gets_u \func_{2n}]\\
        &= \sum_{x \in \set{ 0, 1 }^{2n}} \prob[f(0^{2n}) = 1^{2n} \oplus x \land f(1^{2n}) = x : f \gets_u \func_{2n}]\\
        &= \sum_{x \in \set{ 0, 1 }^{2n}} \frac{1}{2^{2n}} \cdot \frac{1}{2^{2n}}\\
        &= \frac{1}{2^{2n}}
\end{align*}
Now define the distinguisher $D^{h(\cdot)}(\param^n)$
which returns $1$ if $h(0^{2n}) \oplus h(1^{2n}) = 1^{2n}$,
and returns $0$ otherwise.
Then we have
\begin{align*}
    \prob[D^{F_k(\cdot)}(1^n) = 1 : k \gets_u \set{ 0, 1 }^n]
        &= \prob[F_k(0^{2n}) \oplus F_k(1^{2n}) = 1^{2n} : k \gets_u \set{ 0, 1 }^n]\\
        &= \prob[\left(G(k) \oplus 0^{2n}\right) \oplus \left(G(k) \oplus 1^{2n}\right) = 1^{2n} : k \gets_u \set{ 0, 1 }^n]\\
        &= \prob[0^{2n} \oplus 1^{2n} = 1^{2n}] = 1
\end{align*}
Hence,
\begin{align*}
    |\prob[D^{F_k(\cdot)}(1^n) = 1 : k \gets_u \set{ 0, 1 }^n] - \prob[D^{f(\cdot)}(1^n) = 1 : f \gets_u \func_{2n}]|
        &= 1 - \frac{1}{2^{2n}}
\end{align*}
The final expression is clearly not negligible (it doesn't even go to $0$ as $n \to \infty$, let alone faster than any polynomial),
so $F$ is not a pseudorandom function.
\end{proof}

\textbf{TODO}

\section*{Problem 2}

\textbf{TODO}

\section*{Problem 3}

\textbf{TODO}

\end{document}
