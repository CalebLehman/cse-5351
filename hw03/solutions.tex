\documentclass[12pt]{article}
\usepackage[margin=1in]{geometry}

%%% Packages
% Math
\usepackage{amsmath,amsfonts,amssymb,amsthm}
\numberwithin{equation}{section}
\theoremstyle{plain}
\newtheorem{claim}{Claim}
\newtheorem{corollary}{Corollary}
\newtheorem{booktheoremX}{(Textbook) Theorem}
\newenvironment{booktheorem}[1]{%
\renewcommand\thebooktheoremX{#1}%
\booktheoremX
}{\endbooktheoremX}

% Pseudo code
\usepackage{algorithm, algpseudocode}

% I think this needs to be the last package
\usepackage{hyperref}
%%%

%%% Commands
%% General
\newcommand{\set}[1]{\{ #1 \}}

%% Probability commands
\DeclareMathOperator*{\prob}{Pr}
\newcommand{\given}{\mid}

%% Crypto commands
\newcommand{\ppt}{\algo{PPT}}
\newcommand{\func}{\algo{Func}}
\newcommand{\ctexts}{\mathcal{C}}
\newcommand{\ctext}{\algo{C}}
\newcommand{\ptexts}{\mathcal{M}}
\newcommand{\ptext}{\algo{M}}
\newcommand{\keys}{\mathcal{K}}
\newcommand{\key}{\algo{K}}
\newcommand{\algo}[1]{\mathsf{#1}}
\newcommand{\adv}{\mathcal{A}}
\newcommand{\advv}{\mathcal{A}'}
\DeclareMathOperator{\negl}{\algo{negl}}
% Private key
\newcommand{\priv}{\Pi}
\newcommand{\gen}{\algo{Gen}}
\newcommand{\enc}{\algo{Enc}}
\newcommand{\dec}{\algo{Dec}}
\newcommand{\param}{1}
\newcommand{\expir}[3]{\algo{PrivK}^{#1}_{{#2},{#3}}}
\newcommand{\eav}{\algo{eav}}
\newcommand{\cpa}{\algo{cpa}}
%%%

\title{%
CSE 5351 Spring 2020\\
Homework 3
}
\author{%
Caleb Lehman
(\href{mailto:lehman.346@osu.edu}{\texttt{lehman.346@osu.edu}})
}
\date{%
}

\begin{document}

\maketitle

\section*{Problem 1}

Let $G$ be any pseudorandom generator with expansion factor $l(n) = 2n$.
\begin{claim}
$F: \set{ 0, 1 }^n \times \set{ 0, 1 }^{2n} \to \set{ 0, 1 }^{2n} : (k, x) \mapsto G(k) \oplus x$ is not a pseudorandom function.
\end{claim}
\begin{proof}
Define the distinguisher $D^{h(\cdot)}(\param^n)$
which returns $1$ if $h(0^{2n}) \oplus h(1^{2n}) = 1^{2n}$,
and returns $0$ otherwise.
Let $\func_{2n}$ denote the set of all functions $f : \set{ 0, 1 }^{2n} \to \set{ 0, 1 }^{2n}$.
Then
\begin{align*}
    \prob[D^{f(\cdot)}(1^n) = 1 : f \gets_u \func_{2n}]
        &= \prob[f(0^{2n}) \oplus f(1^{2n}) = 1^{2n} : f \gets_u \func_{2n}]\\
        &= \prob[f(0^{2n}) = 1^{2n} \oplus f(1^{2n}) : f \gets_u \func_{2n}]\\
        &= \sum_{x \in \set{ 0, 1 }^{2n}} \prob[f(0^{2n}) = 1^{2n} \oplus x \land f(1^{2n}) = x : f \gets_u \func_{2n}]\\
        &= \sum_{x \in \set{ 0, 1 }^{2n}} \frac{1}{2^{2n}} \cdot \frac{1}{2^{2n}}\\
        &= \frac{1}{2^{2n}}
\end{align*}
and
\begin{align*}
    \prob[D^{F_k(\cdot)}(1^n) = 1 : k \gets_u \set{ 0, 1 }^n]
        &= \prob[F_k(0^{2n}) \oplus F_k(1^{2n}) = 1^{2n} : k \gets_u \set{ 0, 1 }^n]\\
        &= \prob[\left(G(k) \oplus 0^{2n}\right) \oplus \left(G(k) \oplus 1^{2n}\right) = 1^{2n} : k \gets_u \set{ 0, 1 }^n]\\
        &= \prob[0^{2n} \oplus 1^{2n} = 1^{2n}] = 1
\end{align*}
Hence,
\begin{align*}
    |\prob[D^{F_k(\cdot)}(1^n) = 1 : k \gets_u \set{ 0, 1 }^n] - \prob[D^{f(\cdot)}(1^n) = 1 : f \gets_u \func_{2n}]|
        &= 1 - \frac{1}{2^{2n}}
\end{align*}
The final expression is clearly not negligible (it doesn't even go to $0$ as $n \to \infty$, let alone faster than any polynomial),
and $D$ is clearly a $\ppt$ algorithm,
so $F$ is not a pseudorandom function.
\end{proof}

\section*{Problem 2}

\begin{enumerate}

\item[(a)] This encryption scheme is not EAV-secure.
In particular, since the pseudorandom generator $G$ is known to all,
the adversary can always just use the random string $r$, which is included in the ciphertext,
to compute $G(r)$ and recover the plaintext.
Specifically, the adversary can use the following algorithm, which is always successful, for the $\expir{\eav}{\adv}{\Pi}$ experiment:
\begin{algorithm}[H]
\begin{algorithmic}
\Procedure{Attack}{}
    \State $m_0 \gets 0^{n+1}$, $m_1 \gets 1^{n+1}$ (or any two disctinct messages)
    \State Send $m_0$, $m_1$
    \State Receive $c \gets \langle r, c' \rangle$
    \If{$G(r) \oplus c' = m_0$}
        \State \Return 0
    \ElsIf{$G(r) \oplus c' = m_1$}
        \State \Return 1
    \Else
        \State \Return $\bot$
    \EndIf
\EndProcedure
\end{algorithmic}
\end{algorithm}
Hence, this encryption scheme \textbf{is not EAV-secure}.
It follows that this scheme \textbf{is also not CPA-secure},
since that is a stricly stronger measure of security.

\item[(b)] This encryption scheme is EAV-secure.
To see this, define $G': \set{ 0, 1 }^n \to \set{ 0, 1 }^n$ by $G'(k) = F_k(0^n)$.
By Theorem 3.18 in the text, the encryption scheme $c := G(k) \oplus m$ is EAV-secure for any pseudorandom generator $G$,
so we need only show that $G'$ is indeed a pseudorandom generator.
Let $D'$ be an arbitrary $\ppt$ algorithm and define $D^{h(\cdot)}$ by
\begin{equation*}
    D^{h(\cdot)}(1^n) =
    \begin{cases}
        1, &\text{if }D'(h(0^n)) = 1\\
        0, &\text{otherwise}
    \end{cases}
\end{equation*}
Clearly $D$ is still $\ppt$.
Since $F$ is a pseudorandom function,
there is a negligible function $\negl$ such that
\begin{align*}
    \negl(n) 
        &> |\prob[D^{F_k(\cdot)}(1^n) = 1 : k \gets_u \set{ 0, 1 }^n] - \prob[D^{f(\cdot)}(1^n) = 1 : f \gets_u \func_n]|\\
        &= |\prob[D'(F_k(0^n)) = 1 : k \gets_u \set{ 0, 1 }^n] - \prob[D^{f(\cdot)}(1^n) = 1 : f \gets_u \func_n]|\\
        &= |\prob[D'(G'(k)) = 1 : k \gets_u \set{ 0, 1 }^n] - \prob[D^{f(\cdot)}(1^n) = 1 : f \gets_u \func_n]|\\
        &= |\prob[D'(G'(k)) = 1 : k \gets_u \set{ 0, 1 }^n] - \prob[D'(r) = 1 : r \gets_u \set{ 0, 1 }^n]|
\end{align*}
where the final equality follows from the fact that
$\{r\}_{r \gets_u \set{0,1}^n}$ and $\{f(0^n)\}_{f \gets_u \func_n}$
have identical distributions.
Hence, $G'$ is a pseudorandom generator and this encryption scheme \textbf{is EAV-secure}.

However, this scheme \textbf{is not CPA-secure}, since the encryption is deterministic once a key is selected.
In particular, the following algorithm for the $\expir{\cpa}{\adv}{\Pi}$ experiment is always successful:
\begin{algorithm}[H]
\begin{algorithmic}
\Procedure{Attack}{}
    \State $m_0 \gets 0^{n+1}$, $m_1 \gets 1^{n+1}$ (or any two distinct messages)
    \State Send $m_0$, $m_1$
    \State Receive $c \gets \enc_k(m_b)$
    \State Receive an oracle $h(\cdot)$ for $\enc_k(\cdot)$
    \If{$h(m_0) = c$}
        \State \Return 0
    \ElsIf{$h(m_1) = c$}
        \State \Return 1
    \Else
        \State \Return $\bot$
    \EndIf
\EndProcedure
\end{algorithmic}
\end{algorithm}

\item[(c)] This encryption scheme \textbf{is CPA-secure (and therefore EAV-secure)}.
Indeed, since $F$ is a pseudorandom function,
the corresponding block-cipher in CTR mode is CPA-secure (Theorem 3.32 in the text).
This encryption scheme is identical, except the message space has been restricted to messages with a length equal to exactly 2 block lengths,
so it is CPA-secure as well.

\end{enumerate}

\section*{Problem 3}

Regardless of the length of the plaintext,
we must pad it with something, otherwise it would be impossible to distinguish
between, for example, a 1024-bit message ending in a \texttt{1}, and a 1023-bit message that was padded to 1024 bits with a \texttt{1}.
Hence, we would pad the original 1024-bit message with a single \texttt{1} followed 127 \texttt{0}s to get a 1152-bit message,
that is, a $1152 / 128 = 9$-block message.
Since the cipher is in CBC mode, the encrypted ciphertext would have an extra block to pass the initialization vector.
Hence, the total length of the ciphertext would be \textbf{10 blocks} $=$ \textbf{1280 bits}.

\end{document}
