\documentclass[12pt]{article}
\usepackage[margin=1in]{geometry}

%%% Packages
% Math
\usepackage{amsmath,amsfonts,amssymb,amsthm}
\numberwithin{equation}{section}
\theoremstyle{plain}
\newtheorem{claim}{Claim}
\newtheorem{corollary}{Corollary}
\newtheorem{booktheoremX}{(Textbook) Theorem}
\newenvironment{booktheorem}[1]{%
\renewcommand\thebooktheoremX{#1}%
\booktheoremX
}{\endbooktheoremX}

% Pseudo code
%\usepackage{algorithm, algpseudocode}

% I think this needs to be the last package
\usepackage{hyperref}
%%%

%%% Commands
%% General
\newcommand{\set}[1]{\{ #1 \}}

%% Probability commands
\DeclareMathOperator*{\prob}{Pr}
\newcommand{\given}{\mid}

%% Crypto commands
\newcommand{\ctexts}{\mathcal{C}}
\newcommand{\ctext}{\algo{C}}
\newcommand{\ptexts}{\mathcal{M}}
\newcommand{\ptext}{\algo{M}}
\newcommand{\keys}{\mathcal{K}}
\newcommand{\key}{\algo{K}}
\newcommand{\algo}[1]{\mathsf{#1}}
\newcommand{\adv}{\mathcal{A}}
\newcommand{\advv}{\mathcal{A}'}
\DeclareMathOperator{\negl}{negl}
% Private key
\newcommand{\priv}{\Pi}
\newcommand{\gen}{\algo{Gen}}
\newcommand{\enc}{\algo{Enc}}
\newcommand{\dec}{\algo{Dec}}
\newcommand{\param}{1}
\newcommand{\expir}[3]{\algo{PrivK}^{#1}_{{#2},{#3}}}
\newcommand{\eav}{\algo{eav}}
%%%

\title{%
CSE 5351 Spring 2020\\
Homework 2
}
\author{%
Caleb Lehman
(\href{mailto:lehman.346@osu.edu}{\texttt{lehman.346@osu.edu}})
}
\date{%
}

\begin{document}

\maketitle

\textbf{Note}: Throughout this assignment,
we use $a, \ldots, z$ and $0, \ldots, 25$ interchangeably (e.g. ``$a + 1 = b$'', ``$z + 3 = c$'').

\section*{Problem 1}

\begin{enumerate}
    \item[(a)] Fix any $m \in \ptexts$ and set $k_1 = (10 - m) \mod 26$, $k_2 = (5 - m) \mod 26$.
    Then we have
    \begin{align*}
        \prob[\enc_{k_1}(m) = 10]
            &= \frac{1}{2} \prob[10 = m + (10 - m) \bmod 26]\\
            &\mathrel{\phantom{=}} \mathrel{+} \frac{1}{2} \prob[10 = m + (10 - m) + 5 \bmod 26]\\
            &= \frac{1}{2} + 0 = \frac{1}{2},\\
        \prob[\enc_{k_2}(m) = 10]
            &= \frac{1}{2} \prob[10 = m + (5 - m) \bmod 26]\\
            &\mathrel{\phantom{=}} \mathrel{+} \frac{1}{2} \prob[10 = m + (5 - m) + 5 \bmod 26]\\
            &= 0 + \frac{1}{2} = \frac{1}{2},
    \end{align*}
    and for $k \in \keys \setminus \set{ k_1, k_2 }$, $\prob[\enc_{k}(m) = 10] = 0$.
    It follows that
    \begin{align*}
        \prob[\enc_{\key}(m) = 10]
            &= \sum_{k \in \keys}{ \prob[\key = k] \cdot \prob[\enc_{k}(m) = 10] }\\
            &= \frac{1}{26} \sum_{k \in \keys} { \prob[\enc_{k}(m) = 10] }\\
            &= \frac{1}{26} ( \frac{1}{2} + \frac{1}{2} + \sum_{k \neq k_1, k_2} { \prob[\enc_{k}(m) = 10] } )\\
            &= \frac{1}{26} ( 1 + 0 ) = \frac{1}{26}
    \end{align*}
    Note that the particular value of the ciphertext $c = 10$ did not matter.
    In particular $\prob[\enc_{\key}(m) = c] = \frac{1}{26}$
    for all $m, c \in \set{ 0, \ldots, 25 }$.

    \item[(b)] Using the result above, we have
    \begin{align*}
        \prob[\enc_{\key}(\ptext) = 10]
            &= \sum_{m \in \ptexts}{ \prob[\ptext = m] \cdot \prob[\enc_{\key}(m) = 10] }\\
            &= \frac{1}{26} \sum_{m \in \ptexts}{ \prob[\ptext = m] }\\
            &= \frac{1}{26}
    \end{align*}
    Again, note that the particular value of the ciphertext $c = 10$ did not matter.
    In particular $\prob[\enc_{\key}(\ptext) = c] = \frac{1}{26}$
    for all $c \in \set{ 0, \ldots, 25 }$.
\end{enumerate}

\section*{Problem 2}

% define local period command
\newcommand{\period}{\algo{T}}

Let $\period$ be the random variable corresponding to the chosen period of the key.
Then we have
\begin{align*}
    \prob[\key = \texttt{a}]
        &= \prob[\period = 1] \cdot \prob[\key = \texttt{a} \given \period = 1] = \frac{1}{3} \cdot \frac{1}{26}\\
    \prob[\key = \texttt{ab}]
        &= \prob[\period = 2] \cdot \prob[\key = \texttt{ab} \given \period = 2] = \frac{1}{3} \cdot \frac{1}{26^2}\\
    \prob[\key = \texttt{abc}]
        &= \prob[\period = 3] \cdot \prob[\key = \texttt{abc} \given \period = 3] = \frac{1}{3} \cdot \frac{1}{26^3}\\
\end{align*}

% undefine local period command
\let\period\undefined

\section*{Problem 3}

\begin{enumerate}
    \item[(a)] Fix some $k \in \keys$.
    If $|k| = 1$,
    then the first two characters of $\enc_{k}(\texttt{aab})$ will be the same,
    since the first two characters of \texttt{aab} are the same.
    If $|k| \geq 2$,
    then the first two characters of $\enc_{k}(\texttt{aab})$ will be the same
    precisely when the first two characters of $k$ are the same.
    In summary, the set of keys for which $A(\texttt{aab}, \texttt{abb}, \enc_{k}(\texttt{aab})) = 0$ is
    \begin{gather*}
        \keys_0 = \keys_{01} \sqcup \keys_{02} \sqcup \keys_{03},
    \end{gather*}
    where
    \begin{align*}
        \keys_{01} &= \set{ a, \ldots, z }\\
        \keys_{02} &= \set{ \sigma \sigma \mid \sigma \in \set{ a, \ldots, z } }\\
        \keys_{03} &= \set{ \sigma \sigma \sigma' \mid \sigma, \sigma' \in \set{ a, \ldots, z } }
    \end{align*}

    \item[(b)] It is easier to first determine the keys for which $A$ fails,
    that is, $A(\texttt{aab}, \texttt{abb}, \enc_{k}(\texttt{abb})) = 0$.
    In other words, we will first determine the keys for which
    the first two characters of $\enc_{k}(\texttt{abb})$ match,
    and then take the complement.

    Fix some $k \in \keys$.
    If $|k| = 1$,
    then the first two characters of $\enc_{k}(\texttt{abb})$ cannot match,
    since the first two characters of \texttt{abb} are different.
    If $|k| \geq 2$,
    then the first two characters of $\enc_{k}(\texttt{abb})$ will match
    precisely when the second character of $k$ is one greater than the first (modulo 26).
    Taking complements of the sets described above,
    we can describe the set of keys for which $A(\texttt{aab}, \texttt{abb}, \enc_{k}(\texttt{abb})) = 1$ is
    \begin{gather*}
        \keys_1 = \keys_{11} \sqcup \keys_{12} \sqcup \keys_{13},
    \end{gather*}
    where
    \begin{align*}
        \keys_{11} &= \set{ a, \ldots, z } \setminus \set{ }\\
        \keys_{12} &= \set{ a, \ldots, z }^2 \setminus \set{ \sigma(\sigma+1) \mid \sigma \in \set{ a, \ldots, z } }\\
        \keys_{13} &= \set{ a, \ldots, z }^3 \setminus \set{ \sigma(\sigma+1)\sigma' \mid \sigma, \sigma' \in \set{ a, \ldots, z } }
    \end{align*}
\end{enumerate}

\section*{Problem 4}

\newcommand{\bit}{\algo{B}}

Using the sets $\keys_0, \keys_{01}, \keys_{02}, \keys_{03}, \keys_1, \keys_{11}, \keys_{12}, \keys_{13}$ defined in problem 3,
we have
\begin{align*}
    \sum_{k \in \keys}&{ \prob[\key = k] \cdot \prob[A(m_0, m_1, \enc_k(m_0)) = 0] }\\
        &= \sum_{k \in \keys_0} \prob[\key = k]\\
        &= \sum_{k \in \keys_{01}}{ \prob[\key = k] } + \sum_{k \in \keys_{02}}{ \prob[\key = k] } + \sum_{k \in \keys_{03}}{ \prob[\key = k] }\\
        &= \frac{|\keys_{01}|}{3 \cdot 26} + \frac{|\keys_{02}|}{3 \cdot 26^2} + \frac{|\keys_{03}|}{3 \cdot 26^3}\\
        &= \frac{26}{3 \cdot 26} + \frac{26}{3 \cdot 26^2} + \frac{26^2}{3 \cdot 26^3}\\
        &= \frac{1}{3} + \frac{2}{3 \cdot 26}\\
    \sum_{k \in \keys}&{ \prob[\key = k] \cdot \prob[A(m_0, m_1, \enc_k(m_1)) = 1] }\\
        &= \sum_{k \in \keys_1} \prob[\key = k]\\
        &= \sum_{k \in \keys_{11}}{ \prob[\key = k] } + \sum_{k \in \keys_{12}}{ \prob[\key = k] } + \sum_{k \in \keys_{13}}{ \prob[\key = k] }\\
        &= \frac{|\keys_{11}|}{3 \cdot 26} + \frac{|\keys_{12}|}{3 \cdot 26^2} + \frac{|\keys_{13}|}{3 \cdot 26^3}\\
        &= \frac{26 - 0}{3 \cdot 26} + \frac{26^2 - 26}{3 \cdot 26^2} + \frac{26^3 - 26^2}{3 \cdot 26^3}\\
        &= \frac{1}{3} + \frac{25}{3 \cdot 26} + \frac{25}{3 \cdot 26}\\
        &= \frac{1}{3} + \frac{50}{3 \cdot 26}
\end{align*}
Let $\bit$ be the random variable corresponding to which message is chosen.
Then
\begin{align*}
    \prob[\expir{\eav}{A}{\priv}(m_0, m_1) = 1]
        &= \sum_{b \in \set{ 0, 1 }}{ \prob[\bit = b] \cdot \prob[A(m_0, m_1, \enc_{\key}(m_b)) = b] }\\
        &= \frac{1}{2} \sum_{k \in \keys}{ \prob[\key = k] \cdot \prob[A(m_0, m_1, \enc_{k}(m_0)) = 0] }\\
        &\mathrel{\phantom{=}} \mathrel{+} \frac{1}{2} \sum_{k \in \keys}{ \prob[\key = k] \cdot \prob[A(m_0, m_1, \enc_{k}(m_1)) = 1] }\\
        &= \frac{1}{2} ( \frac{1}{3} + \frac{2}{3 \cdot 26} ) + \frac{1}{2} ( \frac{1}{3} + \frac{50}{3 \cdot 26} )\\
        &= \frac{1}{2} (\frac{2}{3} + \frac{52}{3 \cdot 26} )\\
        &= \frac{2}{3}
\end{align*}

\let\bit\undefined

\end{document}
