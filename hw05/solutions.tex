\documentclass[12pt]{article}
\usepackage[margin=1in]{geometry}

%%% Packages
% Math
\usepackage{amsmath,amsfonts,amssymb,amsthm}
\numberwithin{equation}{section}
\theoremstyle{plain}
\newtheorem{claim}{Claim}
\newtheorem{corollary}{Corollary}
\newtheorem{booktheoremX}{(Textbook) Theorem}
\newenvironment{booktheorem}[1]{%
\renewcommand\thebooktheoremX{#1}%
\booktheoremX
}{\endbooktheoremX}

% Pseudo code
\usepackage{algorithm, algpseudocode}

% I think this needs to be the last package
\usepackage{hyperref}
%%%

%%% Commands
%% General
\newcommand{\set}[1]{\{ #1 \}}
\mathchardef\mhyphen="2D 

%% Probability commands
\DeclareMathOperator*{\prob}{Pr}
\newcommand{\given}{\mid}

%% Crypto commands
\newcommand{\ppt}{\algo{PPT}}
\newcommand{\func}{\algo{Func}}
\newcommand{\ctexts}{\mathcal{C}}
\newcommand{\ctext}{\algo{C}}
\newcommand{\ptexts}{\mathcal{M}}
\newcommand{\ptext}{\algo{M}}
\newcommand{\keys}{\mathcal{K}}
\newcommand{\key}{\algo{K}}
\newcommand{\algo}[1]{\mathsf{#1}}
\newcommand{\adv}{\mathcal{A}}
\newcommand{\advv}{\mathcal{A}'}
\DeclareMathOperator{\negl}{\algo{negl}}
% Private key
\newcommand{\priv}{\Pi}
\newcommand{\gen}{\algo{Gen}}
\newcommand{\enc}{\algo{Enc}}
\newcommand{\dec}{\algo{Dec}}
\newcommand{\mac}{\algo{Mac}}
\newcommand{\param}{1}
\newcommand{\privexpir}[3]{\algo{PrivK}^{#1}_{{#2},{#3}}}
\newcommand{\macexpir}[2]{\algo{MACForge}_{{#1},{#2}}}
\newcommand{\eav}{\algo{eav}}
\newcommand{\cpa}{\algo{cpa}}
%%%

\title{%
CSE 5351 Spring 2020\\
Homework 5
}
\author{%
Caleb Lehman
(\href{mailto:lehman.346@osu.edu}{\texttt{lehman.346@osu.edu}})
}
\date{%
}

\begin{document}

\maketitle

\section*{Problem 1}

\begin{claim}
Modify basic $\algo{CBC}\mhyphen \algo{MAC}$ as follows:
For key $k \in \set{ 0, 1 }^n$ and message $m \in \set{ 0, 1 }^{nq}$,
\begin{itemize}
    \item parse $m$ as $m = (m_1, \ldots, m_q)$
    \item let $t_0 \gets_u \set{ 0, 1 }^n$,
    $t_i = F_k(m_i \oplus t_{i-1})$ for $1 \leq i \leq q$
    \item output $\langle t_0, t_q \rangle$ as the tag
\end{itemize}
Then this modified fixed-length $\mac$ scheme is \textbf{not} secure.
\end{claim}
\begin{proof}
Call the modified encryption scheme $\Pi$
and suppose we only have access to an oracle $\mac_k(\cdot)$.
We can get the valid pair $(m_0, \langle t_0, t \rangle) = \mac_k(m_0)$ for the message $m_0 = 0^n$,
where $t = F_k(m_0 \oplus t_0) = F_k(t_0)$ by the definition of $\algo{CBC}$ mode.
Then for any $m \in \set{ 0, 1 }^n$, we have
$F_k((t_0 \oplus m) \oplus m) = F_k(t_0) = t$,
so the pair $(m, \langle t_0 \oplus m, t \rangle)$ is valid.
We can use this attack to develop an adversary:
\begin{algorithm}[H]
\begin{algorithmic}
\Procedure{Adversary}{}
    \State Send the message $0^n$ to the $\mac$ oracle
    \State Receive the tag $\langle t_0, t \rangle$
    \State \Return $(1^n, \langle 1^n \oplus t_0, t \rangle)$
\EndProcedure
\end{algorithmic}
\end{algorithm}
Since $1^n$ was not the message we sent to the oracle,
this adversary always wins the experiment $\macexpir{\adv}{\Pi}(\cdot)$.
It follows that $\Pi$ is not secure.
\end{proof}

\newpage
\section*{Problem 2}

\begin{claim}
Consider the following modification of the basic fixed-length $\algo{CBC}\mhyphen \algo{MAC}$:
Given a message $m = (m_1, \ldots, m_q)$,
append $|m|$ (the $n$-bit representation of the length)
and then apply basic fixed-length $\algo{CBC}\mhyphen \algo{MAC}$.
Then this modified $\mac$ scheme is \textbf{not} secure for arbitrary-length messages.
\end{claim}
\begin{proof}
Let $l_1$, $l_3$ denote the $n$-bit representations of 1 and 3, that is,
the lengths of single-block and a three-block messages, respectively.
Send $0^n$, $1^n$ to the oracle and let
$t_1 = \algo{basic}\mhyphen \algo{CBC}\mhyphen \algo{MAC}_k(0^n \| l_1)$,
$t_2 = \algo{basic}\mhyphen \algo{CBC}\mhyphen \algo{MAC}_k(1^n \| l_1)$
be the tags it returns.
Then send $0^n \| l_1 \| 0^n$ to the oracle and let
$t_3 = \algo{basic}\mhyphen \algo{CBC}\mhyphen \algo{MAC}_k(0^n \| l_1 \| 0^n \| l_3)$
be the tag it returns.
Note that
\begin{align*}
    \algo{basic}\mhyphen \algo{CBC}\mhyphen \algo{MAC}_k(1^n \| l_1 \| (t_1 \oplus t_2) \| l_3)
        &= F_k(l_3 \oplus F_k((t_1 \oplus t_2) \oplus F_k(l_1 \oplus F_k(1^n))))\\
        &= F_k(l_3 \oplus F_k((t_1 \oplus t_2) \oplus \algo{basic}\mhyphen \algo{CBC}\mhyphen \algo{MAC}_k(1^n \| l_1)))\\
        &= F_k(l_3 \oplus F_k((t_1 \oplus t_2) \oplus t_2))\\
        &= F_k(l_3 \oplus F_k(0^n \oplus t_1))\\
        &= F_k(l_3 \oplus F_k(0^n \oplus \algo{basic}\mhyphen \algo{CBC}\mhyphen \algo{MAC}_k(0^n \| l_1)))\\
        &= F_k(l_3 \oplus F_k(0^n \oplus F_k(l_1 \oplus F_k(0^n))))\\
        &= t_3
\end{align*}
so $(1^n \| l_1 \| (t_1 \oplus t_2), t_3)$ is a valid pair as well.
Since $1^n \| l_1 \| (t_1 \oplus t_2)$ is not one of the messages we passed to the oracle,
we can use this exploit to design the following adversary:
\begin{algorithm}[H]
\begin{algorithmic}
\Procedure{Adversary}{}
    \State Send the messages $0^n$, $1^n$, and $0^n \| l_1 \| 0^n$ to the $\mac$ oracle
    \State Receive the appropriate tags $t_1$, $t_2$, and $t_3$ from the oracle
    \State \Return $(1^n \| l_1 \| (t_1 \oplus t_2), t_3)$
\EndProcedure
\end{algorithmic}
\end{algorithm}
This adversary always wins the experiment $\macexpir{\adv}{\Pi}(\cdot)$,
so this scheme is not secure for arbitrary-length messages.
\end{proof}

\newpage
\section*{Problem 3}

\begin{claim}
Let $F$ be a pseudorandom function
and construct a fixed-length $\mac$ scheme for messages as follows:
For key $k \in \set{ 0, 1 }^n$ and message $m \in \set{ 0, 1 }^{2n}$,
\begin{itemize}
    \item parse $m$ as $m = m_1 \| m_2$, where $|m_1| = |m_2| = n$
    \item output $F_k(m_1) \| F_k(F_k(m_2))$ as the tag
\end{itemize}
Then this fixed-length $\mac$ scheme is \textbf{not} secure against chosen-message attacks.
\end{claim}
\begin{proof}
Note that the first half of the tag depends only on the first half of
the message and the second half of the tag depends only on the second
half of the message.
In particular, by querying the $\mac$ oracle on the messages
$m = m_1 \| m_2$ and $n = n_1 \| n_2$,
we get the tags
$t^m = t^m_1 \| t^m_2 = F_k(m_1) \| F_k(F_k(m_2))$ and $t^n = t^n_1 \| t^n_2 = F_k(n_1) \| F_k(F_k(n_2))$,
which we can mix to make the valid pair
$(m_1 \| n_2, t^m_1 \| t^n_2)$.
This observation allows us to develop the following adversary:
\begin{algorithm}[H]
\begin{algorithmic}
\Procedure{Adversary}{}
    \State Send $m_0 = 0^n \| 0^n$ to the $\mac$ oracle
    \State Receive tag and set $t_0 \gets$ the \textit{first} half of the bits of the tag
    \State Send $m_1 = 1^n \| 1^n$ to the $\mac$ oracle
    \State Receive tag and set $t_1 \gets$ the \textit{second} half of the bits of the tag
    \State \Return $(0^n \| 1^n, t_0 \| t_1)$
\EndProcedure
\end{algorithmic}
\end{algorithm}
The message $0^n \| 1^n$ is not one of the two messages the adversary sends to the oracle,
so this adversary always wins the experiment $\macexpir{\adv}{\Pi}(\cdot)$.
Hence, this scheme is not secure.
\end{proof}

\end{document}
