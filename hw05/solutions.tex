\documentclass[12pt]{article}
\usepackage[margin=1in]{geometry}

%%% Packages
% Math
\usepackage{amsmath,amsfonts,amssymb,amsthm}
\numberwithin{equation}{section}
\theoremstyle{plain}
\newtheorem{claim}{Claim}
\newtheorem{corollary}{Corollary}
\newtheorem{booktheoremX}{(Textbook) Theorem}
\newenvironment{booktheorem}[1]{%
\renewcommand\thebooktheoremX{#1}%
\booktheoremX
}{\endbooktheoremX}

% Pseudo code
\usepackage{algorithm, algpseudocode}

% I think this needs to be the last package
\usepackage{hyperref}
%%%

%%% Commands
%% General
\newcommand{\set}[1]{\{ #1 \}}
\mathchardef\mhyphen="2D 

%% Probability commands
\DeclareMathOperator*{\prob}{Pr}
\newcommand{\given}{\mid}

%% Crypto commands
\newcommand{\ppt}{\algo{PPT}}
\newcommand{\func}{\algo{Func}}
\newcommand{\ctexts}{\mathcal{C}}
\newcommand{\ctext}{\algo{C}}
\newcommand{\ptexts}{\mathcal{M}}
\newcommand{\ptext}{\algo{M}}
\newcommand{\keys}{\mathcal{K}}
\newcommand{\key}{\algo{K}}
\newcommand{\algo}[1]{\mathsf{#1}}
\newcommand{\adv}{\mathcal{A}}
\newcommand{\advv}{\mathcal{A}'}
\DeclareMathOperator{\negl}{\algo{negl}}
% Private key
\newcommand{\priv}{\Pi}
\newcommand{\gen}{\algo{Gen}}
\newcommand{\enc}{\algo{Enc}}
\newcommand{\dec}{\algo{Dec}}
\newcommand{\mac}{\algo{Mac}}
\newcommand{\param}{1}
\newcommand{\privexpir}[3]{\algo{PrivK}^{#1}_{{#2},{#3}}}
\newcommand{\macexpir}[2]{\algo{MACForge}_{{#1},{#2}}}
\newcommand{\eav}{\algo{eav}}
\newcommand{\cpa}{\algo{cpa}}
%%%

\title{%
CSE 5351 Spring 2020\\
Homework 5
}
\author{%
Caleb Lehman
(\href{mailto:lehman.346@osu.edu}{\texttt{lehman.346@osu.edu}})
}
\date{%
}

\begin{document}

\maketitle

\section*{Problem 1}

\begin{claim}
Modify basic $\algo{CBC}\mhyphen \algo{MAC}$ as follows:
For key $k \in \set{ 0, 1 }^n$ and message $m \in \set{ 0, 1 }^{nq}$,
\begin{itemize}
    \item parse $m$ as $m = (m_1, \ldots, m_q)$
    \item let $t_0 \gets_u \set{ 0, 1 }^n$,
    $t_i = F_k(m_i \oplus t_{i-1})$ for $1 \leq i \leq q$
    \item output $\langle t_0, t_q \rangle$ as the tag
\end{itemize}
Then this modified fixed-length $\mac$ scheme is \textbf{not} secure.
\end{claim}
\begin{proof}
Call the modified encryption scheme $\Pi$
and suppose we only have access to an oracle $\mac_k(\cdot)$.
We can get the valid pair $(m_0, \langle t_0, t \rangle) = \mac_k(m_0)$ for the message $m_0 = 0^n$,
where $t = F_k(m_0 \oplus t_0) = F_k(t_0)$ by the definition of $\algo{CBC}$ mode.
Then for any $m \in \set{ 0, 1 }^n$, we have
$F_k((t_0 \oplus m) \oplus m) = F_k(t_0) = t$,
so the pair $(m, \langle t_0 \oplus m, t \rangle)$ is valid.
By choosing $m \neq 0^n$, this process easily defines an adversary, $\adv$,
that always wins experiment $\macexpir{\adv}{\Pi}(\cdot)$.
It follows that $\Pi$ is not secure.
\end{proof}

\section*{Problem 2}

\textbf{TO-DO}

\section*{Problem 3}

\begin{claim}
Let $F$ be a pseudorandom function
and construct a fixed-length $\mac$ scheme for messages as follows:
For key $k \in \set{ 0, 1 }^n$ and message $m \in \set{ 0, 1 }^{2n}$,
\begin{itemize}
    \item parse $m$ as $m = m_1 \| m_2$, where $|m_1| = |m_2| = n$
    \item output $F_k(m_1) \| F_k(F_k(m_2))$ as the tag
\end{itemize}
Then this fixed-length $\mac$ scheme is \textbf{not} secure against chosen-message attacks.
\end{claim}
\begin{proof}
Suppose that $m = m_1 \| m_2$, $t = t_1 \| t_2 = \mac_k(m)$,
where $|m_1| = |m_2|$ and $|t_1| = |t_2|$.
Note that $t_1$ depends only on $m_1$ and $t_2$ depends only on $m_2$.
This observation allows us to develop the following adversary:
\end{proof}

\textbf{TO-DO}

\end{document}
