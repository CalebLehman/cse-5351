\documentclass[12pt]{article}
\usepackage[margin=1in]{geometry}

%%% Packages
% Math
\usepackage{amsmath,amsfonts,amssymb,amsthm}
\numberwithin{equation}{section}
\theoremstyle{plain}
\newtheorem{claim}{Claim}
\newtheorem{corollary}{Corollary}
\newtheorem{booktheoremX}{(Textbook) Theorem}
\newenvironment{booktheorem}[1]{%
\renewcommand\thebooktheoremX{#1}%
\booktheoremX
}{\endbooktheoremX}

% Pseudo code
\usepackage{algorithm, algpseudocode}

% I think this needs to be the last package
\usepackage{hyperref}
%%%

%%% Commands
%% General
\newcommand{\set}[1]{\{ #1 \}}

%% Probability commands
\DeclareMathOperator*{\prob}{Pr}
\newcommand{\given}{\mid}

%% Crypto commands
\newcommand{\ctexts}{\mathcal{C}}
\newcommand{\ctext}{\algo{C}}
\newcommand{\ptexts}{\mathcal{M}}
\newcommand{\ptext}{\algo{M}}
\newcommand{\keys}{\mathcal{K}}
\newcommand{\key}{\algo{K}}
\newcommand{\algo}[1]{\mathsf{#1}}
\newcommand{\adv}{\mathcal{A}}
\newcommand{\advv}{\mathcal{A}'}
\DeclareMathOperator{\negl}{negl}
% Private key
\newcommand{\priv}{\Pi}
\newcommand{\gen}{\algo{Gen}}
\newcommand{\enc}{\algo{Enc}}
\newcommand{\dec}{\algo{Dec}}
\newcommand{\param}{1}
\newcommand{\expir}[3]{\algo{PrivK}^{#1}_{{#2},{#3}}}
\newcommand{\eav}{\algo{eav}}
%%%

\title{%
CSE 5351 Spring 2020\\
Homework 1
}
\author{%
Caleb Lehman
(\href{mailto:lehman.346@osu.edu}{\texttt{lehman.346@osu.edu}})
}
\date{%
}

\begin{document}

\maketitle

\section*{Problem 1}

Let $N = 26$, $S = 1 + \ldots + 26 = N(N+1) / 2 = 351$.
For any message $m \in \ptexts$,
there is a unique key $k \in \keys$ such that $0 = \enc_k(m) = (m + k) \bmod N$,
namely $k = (-m) \bmod N$.
It follows that
\begin{equation*}
    \prob[\ctext = 0 \given \ptext = m] = \prob[\key = (-m) \bmod N] = \frac{((-m) \bmod N) + 1}{S}
\end{equation*}
We then have the following computation:
\begin{align*}
    \prob[\ctext = 0]
        &= \sum_{m \in \ptexts}{ \prob[\ptext = m] \cdot \prob[\ctext = 0 \given \ptext = m] } \\
        &= \sum_{m = 0}^{N-1}{ \frac{m + 1}{S} \cdot \frac{((-m \bmod N) + 1}{S} } \\
        &= \frac{1}{S^2} + \sum_{m = 1}^{N-1}{ \frac{m + 1}{S} \cdot \frac{((-m \bmod N) + 1}{S} } \\
        &= \frac{1}{S^2} + \sum_{m = 1}^{N-1}{ \frac{m + 1}{S} \cdot \frac{N+1-m}{S} } \\
        &= \frac{1}{S^2} - \frac{N + 1}{S^2} + \sum_{m = 0}^{N-1}{ \frac{m + 1}{S} \cdot \frac{N+1-m}{S} } \\
        &= -\frac{N}{S^2} + \sum_{m = 0}^{N-1}{ \frac{N}{S^2} (m+1) } + \sum_{m = 0}^{N-1}{ \frac{1 - m^2}{S^2} } \\
        &= -\frac{N}{S^2} + \frac{N}{S} + \frac{N}{S^2} - \frac{1}{S^2} \sum_{m = 0}^{N-1}{ m^2 } \\
        &= \frac{N}{S} - \frac{1}{S^2} \frac{(N-1)N(2N-1)}{6} \\
        &= \frac{2}{3} \frac{N^2 + 6N - 1}{N(N+1)^2}
\end{align*}
Finally, substituting the value $N = 26$, we have $\prob[\ctext = 0] = 277 / 9477 \approx 0.029$


\section*{Problem 2}

\begin{claim}\label{claim:shift-cipher-property}
Consider any message/ciphertext pair produced with a shift cipher,
that is,
any message $m = m_0 \cdots m_n$,
ciphertext $c = c_0 \cdots c_n$,
and pair $(k, N)$
such that $c_i = (m_i + k) \bmod N$ for $i = 0, \ldots, n$.
Then $c_i - c_j \equiv m_i - m_j \mod N$
for all $i, j \in \set{ 0, \ldots, n }$.
\end{claim}
\begin{proof}
Simply observe that $c_i - c_j \equiv (m_i - k) - (m_j - k) \equiv m_i - m_j \mod N$.
\end{proof}

From Claim~(\ref{claim:shift-cipher-property}), we get the following corollaries:
\begin{corollary}
The second character is 1 greater than the first character (modulo $N = 26$)
in any encryption of \texttt{abcd} using a Caesar shift cipher.
\end{corollary}
\begin{corollary}
The second character is 3 greater than the first character (modulo $N = 26$)
in any encryption of \texttt{bedg} using a Caesar shift cipher.
\end{corollary}
It follows that Eve can use the following procedure to determine Bob's password:
\begin{algorithm}[H]
\begin{algorithmic}
\Procedure{BreakBob}{\null}
    \State $c = c_0c_1c_2c_3 \gets$ ciphertext from Bob
    \State $\Delta \gets (c_1 - c_0) \bmod 26$
    \If{$\Delta = 1$}
        \State \Return \texttt{abcd}
    \ElsIf{$\Delta = 3$}
        \State \Return \texttt{bedg}
    \Else
        \State \Return $\bot$
    \EndIf
\EndProcedure
\end{algorithmic}
\end{algorithm}


\section*{Problem 3}

We show that Eve can determine Bob's password for period 3,
but not for periods 2 and 4.
For all of the following, we assume keys are generated uniformly.

\subsection*{Period 2}

Each Vigen\`ere cipher of period 2 corresponds to a key pair $k = (k_0, k_1)$,
where the subsequence $m_im_{i+2}\cdots$
is encrypted using the Caesar shift cipher with key $k_i$.
It is clear that applying the cipher corresponding to $k = (k_0, k_1)$
and then applying the cipher corresponding to $k' = (k'_0, k'_1)$
is equivalent to just applying the cipher $k' \circ k := (k'_0 + k_0, k'_1 + k_1)$.
Furthermore, the keys $\chi = (25, 23)$ and $\chi' = (1, 3)$
encrypt \texttt{abcd} and \texttt{bedg} as each other, that is
\begin{gather*}
    \enc_{\chi}(\texttt{bedg}) = \texttt{abcd}\\
    \enc_{\chi'}(\texttt{abcd}) = \texttt{bedg}
\end{gather*}
Combining these results, we can prove the desired claim:
\begin{claim}\label{claim:period-2-secure}
Assuming Bob uses a Vigen\`ere cipher of period 2,
Eve cannot determine Bob's password from the resulting ciphertext.
In particular, this particular encryption scheme is perfectly secret.
\end{claim}
\begin{proof}
Since all keys are equally likely,
we have $\prob[\key = k] = \prob[\key = k \circ \chi]$ for all $k \in \keys$.
It follows that
\begin{align*}
    \prob[\ctext = c \given \ptext = \texttt{abcd}]
        &= \sum_{k \in \keys}{ \prob[\key = k]\prob[\enc_{k}(\texttt{abcd}) = c] }\\
        &= \sum_{k \in \keys}{ \prob[\key = k]\prob[\enc_{k \circ \chi}(\texttt{bedg}) = c] }\\
        &= \sum_{k \in \keys}{ \prob[\key = k \circ \chi]\prob[\enc_{k \circ \chi}(\texttt{bedg}) = c] }
\end{align*}
Since $f:\keys \to \keys:k \mapsto k \circ \chi$ is a bijection on $\keys$
(with inverse explicitly given by $f':\keys \to \keys: k \mapsto k \circ \chi'$),
we have
\begin{align*}
    \sum_{k \in \keys}{ \prob[\key = k \circ \chi]\prob[\enc_{k \circ \chi}(\texttt{bedg}) = c] }
        &= \sum_{k \in \keys}{ \prob[\key = k]\prob[\enc_{k}(\texttt{bedg}) = c] }\\
        &= \prob[\ctext = c \given \ptext = \texttt{bedg}]
\end{align*}
We conclude that
\begin{equation*}
    \prob[\ctext = c \given \ptext = \texttt{abcd}] = \prob[\ctext = c \given \ptext = \texttt{bedg}]
\end{equation*}
for every ciphertext $c \in C$,
regardless of the probability distribution over $\ptexts = \set{ \texttt{abcd}, \texttt{bedg} }$.
This result is equivalent to perfect secrecy (Lemma (2.4) in the textbook).
\end{proof}

\subsection*{Period 3}

Assuming a period of 3,
the first and fourth characters will be encrypted using the same shift cipher,
so we can apply Claim~(\ref{claim:shift-cipher-property}) to get the following corollaries:
\begin{corollary}
The fourth character is 3 greater than the first character (modulo $N = 26$) in any encryption
of \texttt{abcd} using a Vigen\`ere cipher of period 3.
\end{corollary}
\begin{corollary}
The fourth character is 5 greater than the first character (modulo $N = 26$) in any encryption
of \texttt{bedg} using a Vigen\`ere cipher of period 3.
\end{corollary}
It follows that Eve can use the following procedure to determine Bob's password:
\begin{algorithm}[H]
\begin{algorithmic}
\Procedure{BreakBob}{\null}
    \State $c = c_0c_1c_2c_3 \gets$ ciphertext from Bob
    \State $\Delta \gets (c_3 - c_0) \bmod 26$
    \If{$\Delta = 3$}
        \State \Return \texttt{abcd}
    \ElsIf{$\Delta = 5$}
        \State \Return \texttt{bedg}
    \Else
        \State \Return $\bot$
    \EndIf
\EndProcedure
\end{algorithmic}
\end{algorithm}

\subsection*{Period 4}

Each Vigen\`ere cipher of period 4 corresponds to a key pair $k = (k_0, k_1, k_2, k_3)$,
where the subsequence $m_im_{i+4}\cdots$
is encrypted using the Caesar shift cipher with key $k_i$.
Similarly to the case with period 2, appying the cipher with key $k$
and then applying the cipher with key $k'$
is equivalent to just applying the cipher $k' \circ k$
composed of the sums of the individual shift cipher keys.
Furthermor, note that the keys
$\chi = (25, 23, 25, 23)$ and $\chi' = (1, 3, 1, 3)$
encrypt \texttt{abcd} and \texttt{bedg} as each other, that is
\begin{gather*}
    \enc_{\chi}(\texttt{bedg}) = \texttt{abcd}\\
    \enc_{\chi'}(\texttt{abcd}) = \texttt{bedg}
\end{gather*}
Combining these results, we can prove the following claim:
\begin{claim}\label{claim:period-4-secure}
Assuming Bob uses a Vigen\`ere cipher of period 4,
Eve cannot determine Bob's password from the resulting ciphertext.
In particular, this particular encryption scheme is perfectly secret.
\end{claim}
\begin{proof}
Same proof as for Claim~(\ref{claim:period-2-secure}),
except $\chi$ and $\chi'$ now refer to the 4-tuples above.
\end{proof}


\section*{Problem 4}

The resulting scheme may be an improvement in some sense,
since you are no longer distributing the message in human-readable plaintext.
However, from the standpoint of cryptographic secrecy,
the resulting scheme is not an improvement.
In particular, it is no longer perfectly secret:
\begin{claim}
Fix $n \in \mathbb{N}$ and consider the encrytion scheme $\Pi = (\gen, \enc, \dec)$
with message space $\ptexts = \set{ 0, 1 }^n$ and key space $\keys = \set{0, 1}^n \setminus \set{0^n}$,
where $\enc_k(m) = m \oplus k$,
and $\dec_k(c) = c \oplus k$.
Then $\Pi$ is not perfectly secure.
\end{claim}
\begin{proof}
This follows immediately from textbook Theorem~(\ref{booktheorem:key-space-size}) (re-stated in problem 5),
since $|\keys| = 2^n - 1 < 2^n = |\ptexts|$.

The result can also easily be proved directly.
Take the uniform distribution on $\ptexts$
and consider the message $m_0 = 0^n$ and the ciphertext $c_0 = 0^n$.
Then
\begin{align*}
    \prob[\ctext = c_0]
        &= \sum_{m \in \ptexts, k \in \keys}{ \prob[\ptext = m \wedge \key = k \wedge c = 0^n = m \oplus k] }\\
        &= \sum_{m \in \ptexts}{ \prob[\ptext = m] \cdot \prob[\key = m] }\\
        &= \prob[\ptext = 0^n]\cdot \prob[\key = 0^n] + \sum_{m \neq 0^n}{ \frac{1}{2^n} \cdot \frac{1}{2^n - 1} }\\
        &= 0 + 2^n (\frac{1}{2^n} \cdot \frac{1}{2^n - 1})\\
        &= \frac{1}{2^n - 1}\\
        &> 0,
\end{align*}
but,
\begin{equation*}
    \prob[\ptext = m_0 \given \ctext = c_0] = \prob[\ptext = 0^n \given \ctext = 0^n] = 0,
\end{equation*}
since the only key which encrpyts $m_0$ to $c_0$ is $k = 0^n \not\in \keys$.
Hence, we have found a message $m_0 \in \ptexts$
and a ciphertext $c_0 \in \ctexts$
such that $\prob[\ctext = c_0] > 0$
and
\begin{equation*}
    \prob[\ptext = m_0 \given \ctext = c_0] = 0 < \frac{1}{2^n} = \prob[\ptext = m_0],
\end{equation*}
so $\Pi$ is not perfectly secure.
\end{proof}


\section*{Problem 5}
\begin{enumerate}
    \item[a)] The key space consists of all bijective functions on $\set{ a, b, \ldots, z }$, that is,
    \begin{equation*}
        \keys = \set{ f:\set{a, \ldots, z} \to \set{a, \ldots, z} \mid f \text{ is bijective}}
    \end{equation*}
    It is more likely that we would encode these keys by the result of applying each function to the string $ab\cdots z$,
    that is,
    \begin{align*}
        \keys 
            &= \set{ f(ab\cdots z) \mid f:\set{a, \ldots, z} \to \set{a, \ldots, z} \text{ a bijection} }\\
            &= \set{ \text{permutations of the string } ab\cdots z }
    \end{align*}
    Note that $|\keys| = 26!$.
    
    \item[b)] We first make some observations about message spaces
    for which the mono-alphabetic substitution cipher is perfectly secure
    and collect some theorems:
    \begin{booktheorem}{2.10}\label{booktheorem:key-space-size}
    If $(\gen, \enc, \dec)$ is a perfectly secret encryption scheme with message space
    $\ptexts$ and key space $\keys$, then $|\keys| \geq |\ptexts|$.
    \end{booktheorem}
    \begin{claim}\label{claim:max-message-space}
    Let $\ptexts$ be a message space
    for which the mono-alphabetic substitution cipher is perfectly secure.
    Then $|\ptexts| \leq 26!$.
    \end{claim}
    \begin{proof}
    This follows from textbook Theorem~(\ref{booktheorem:key-space-size})
    and the fact that $|\keys| = 26!$.
    \end{proof}
    \begin{booktheorem}{2.11 (Shannon's theorem)}\label{booktheorem:shannon-theorem}
    Let $(\gen, \enc, \dec)$ be an encryption scheme with message space
    $\ptexts$ and key space $\keys$,
    such that $|\ptexts| = |\ctexts| = |\keys|$
    and each key is generated by $\gen$ with uniform probability.
    Then this scheme is perfectly secret if and only if
    for every $m \in \ptexts$ and every $c \in \ctexts$,
    there is a unique key $k \in \keys$
    such that $\enc_k(m)$ outputs $c$.
    \end{booktheorem}
    We can now describe a maximal message space $\ptexts \subseteq \set{ a, \ldots, z }^*$
    for which the mono-alphabetic substitution cipher provides perfect secrecy:
    \begin{claim}
    The mono-alphabetic substitution cipher is perfectly secure for the message space
    \begin{equation*}
        \ptexts = \set{ \text{permutations of the string } ab\cdots z }
    \end{equation*}
    and this message space is a large a possible while maintaining that property.
    \end{claim}
    \begin{proof}
    We have $\ptexts = \ctexts = \keys = \set{ \text{permutations of the string } ab\cdots z}$.
    In particular, $|\ptexts| = |\ctexts| = |\keys| = 26!$
    and (by hypothesis) each key is equally likely to be generated.
    Now consider any $\pi \in \ptexts$ and $\pi' \in \ctexts$,
    that is,
    any two permutations $\pi, \pi'$
    of the string $ab\cdots z$.
    Given any character $\sigma \in \set{ a, \ldots, z }$,
    there is a unique index $i_{\pi}(\sigma)$
    such that $\sigma$ is the $i_{\pi}(\sigma)$th character of $\pi$.
    Hence, the function
    \begin{equation*}
        f : \set{a,\ldots,z} \to \set{a,\ldots,z}
          : \sigma \mapsto i_{\pi}(\sigma)\text{th character of }\pi'
    \end{equation*}
    is uniquely defined
    (in words, write the two permutations on top of one another and read the substitution from top to bottom).
    Then the string $k = f(a)f(b)\cdots f(z)$
    is a permutation of the string $ab\cdots z$
    and is the unique key such that $\enc_k(\pi) = \pi'$.
    The conditions of Shannon's theorem are satisfied,
    so the mono-alphabetic substitution cipher is perfectly secure for this message space.
    Since $|\ptexts| = 26!$, Claim~(\ref{claim:max-message-space}) gives us
    that this space is maximal while still maintaining perfect secrecy.
    \end{proof}

    Finally, we note that the above message space is not the only maximally-sized message
    space for which the mono-alphabetic substitution cipher is perfectly secure.
    Indeed, by a similar argument, we can show that the maximally-sized message spaces
    are precisely the spaces of the form
    \begin{equation*}
        \ptexts = \set{ f(\omega) \mid f :\set{ a, \ldots, z } \to \set{ a, \ldots, z } \text{ a bijection} },
    \end{equation*}
    where $\omega \in \set{ a, \ldots, z }^*$ is any word with exactly 25 or 26 \emph{distinct} characters
    (allowing for any number of repetitions).
\end{enumerate}
\end{document}
