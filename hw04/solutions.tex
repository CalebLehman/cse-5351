\documentclass[12pt]{article}
\usepackage[margin=1in]{geometry}

%%% Packages
% Math
\usepackage{amsmath,amsfonts,amssymb,amsthm}
\numberwithin{equation}{section}
\theoremstyle{plain}
\newtheorem{claim}{Claim}
\newtheorem{corollary}{Corollary}
\newtheorem{booktheoremX}{(Textbook) Theorem}
\newenvironment{booktheorem}[1]{%
\renewcommand\thebooktheoremX{#1}%
\booktheoremX
}{\endbooktheoremX}

% Pseudo code
\usepackage{algorithm, algpseudocode}

% I think this needs to be the last package
\usepackage{hyperref}
%%%

%%% Commands
%% General
\newcommand{\set}[1]{\{ #1 \}}

%% Probability commands
\DeclareMathOperator*{\prob}{Pr}
\newcommand{\given}{\mid}

%% Crypto commands
\newcommand{\ppt}{\algo{PPT}}
\newcommand{\func}{\algo{Func}}
\newcommand{\ctexts}{\mathcal{C}}
\newcommand{\ctext}{\algo{C}}
\newcommand{\ptexts}{\mathcal{M}}
\newcommand{\ptext}{\algo{M}}
\newcommand{\keys}{\mathcal{K}}
\newcommand{\key}{\algo{K}}
\newcommand{\algo}[1]{\mathsf{#1}}
\newcommand{\adv}{\mathcal{A}}
\newcommand{\advv}{\mathcal{A}'}
\DeclareMathOperator{\negl}{\algo{negl}}
% Private key
\newcommand{\priv}{\Pi}
\newcommand{\gen}{\algo{Gen}}
\newcommand{\enc}{\algo{Enc}}
\newcommand{\dec}{\algo{Dec}}
\newcommand{\param}{1}
\newcommand{\expir}[3]{\algo{PrivK}^{#1}_{{#2},{#3}}}
\newcommand{\eav}{\algo{eav}}
\newcommand{\cpa}{\algo{cpa}}
%%%

\title{%
CSE 5351 Spring 2020\\
Homework 4
}
\author{%
Caleb Lehman
(\href{mailto:lehman.346@osu.edu}{\texttt{lehman.346@osu.edu}})
}
\date{%
}

\begin{document}

\maketitle

\section*{Problem 1}

\begin{claim}
Consider the encryption scheme defined as a variant of CBC-mode encryption,
in which the sender increments the $IV$ by 1 each time a message is encrypted.
Then the resulting scheme is not CPA-secure.
\end{claim}
\begin{proof}
Let $n$ be the block size and suppose an attacker chooses two distinct messages $m_0 \neq m_1$.
The attacker will receive the ciphertext $C = \langle IV, F_k(IV \oplus m_b) \rangle$,
where $k$ is the randomly generated key and $b$ is the randomly generated bit.
The attacker can then ask the oracle for the encryption of $m = (IV \oplus m_0) \oplus (IV + 1)$,
which yields
\begin{align*}
    C'
        &= \langle IV + 1, F_k((IV + 1) \oplus m) \rangle\\
        &= \langle IV + 1, F_k((IV + 1) \oplus (IV \oplus m_0) \oplus (IV + 1)) \rangle\\
        &= \langle IV + 1, F_k(IV \oplus m_0) \rangle
\end{align*}
By comparing with the value $F_k(IV \oplus m_b)$ obtained in the first encryption,
the attacker can determine $b$.
In particular, the following attack is always successful,
so the encryption scheme is not CPA-secure:
\begin{algorithm}[H]
\begin{algorithmic}
\Procedure{Attack}{}
    \State $m_0 \gets 0^n$, $m_1 \gets m_0 \oplus 1^n$ (or any two distinct messages)
    \State Send $m_0$, $m_1$
    \State Receive $C \gets \langle IV, c \rangle$
    \State Compute $m = (IV \oplus m_0) \oplus (IV + 1)$ and pass to oracle
    \State Receive $C' \gets \langle IV + 1, c' \rangle$
    \If{$c = c'$}
        \State \Return 0
    \Else
        \State \Return 1
    \EndIf
\EndProcedure
\end{algorithmic}
\end{algorithm}
\end{proof}

\section*{Problem 2}

Suppose we encrypt the message $m = \langle m_1, m_2, \ldots, m_t \rangle$
and receive the ciphertext $c = \langle c_0, c_1, c_2, \ldots, c_t \rangle$,
where $c_0 = IV$.
Furthermore, suppose that there is a transmission error in $c_q$
for some $0 \leq q \leq t$.
We consider the effect on recovering the plaintext for several different modes of operation.

\subsection*{CTR}

Decryption of the $i$th plaintext block is performed by computing $m_i = c_i \oplus F_k(IV + i) = c_i \oplus F_k(c_0 + i)$.
The only ciphertext blocks on which this depends are $c_0$ and $c_i$.
As a result, there are two cases:
\begin{equation*}
\begin{cases}
q = 0, &\text{all plaintext blocks will be recovered \textit{incorrectly}}\\
0 < q \leq t, &\text{only }m_q\text{ will be recovered \textit{incorrectly}}
\end{cases}
\end{equation*}

\subsection*{OFB}

Decryption of the $i$th plaintext block is performed by computing $m_i = c_i \oplus F_k^i(IV) = c_i \oplus F_k^i(c_0)$,
where $F_k^i$ denotes the application of $F_k$ $i$ times.
The only ciphertext blocks on which this depends are $c_0$ and $c_i$.
As a result, there are two cases:
\begin{equation*}
\begin{cases}
q = 0, &\text{all plaintext blocks will be recovered \textit{incorrectly}}\\
0 < q \leq t, &\text{only }m_q\text{ will be recovered \textit{incorrectly}}
\end{cases}
\end{equation*}

\subsection*{CFB}

Decryption of the $i$th plaintext block is performed by computing $m_i = c_i \oplus F_k(c_{i-1})$.
The only ciphertext blocks on which this depends are $c_{i-1}$ and $c_i$.
As a result, there are three cases:
\begin{equation*}
\begin{cases}
q = 0, &\text{only }m_1\text{ will be recovered \textit{incorrectly}}\\
0 < q < t, &\text{only }m_q\text{ and }m_{q+1}\text{ will be recovered \textit{incorrectly}}\\
q = t, &\text{only }m_t\text{ will be recovered \textit{incorrectly}}
\end{cases}
\end{equation*}

\subsection*{CBC}

Decryption of the $i$th plaintext block is performed by computing $m_i = c_{i-1} \oplus F_k^{-1}(c_i)$.
The only ciphertext blocks on which this depends are $c_{i-1}$ and $c_i$.
As a result, there are three cases:
\begin{equation*}
\begin{cases}
q = 0, &\text{only }m_1\text{ will be recovered \textit{incorrectly}}\\
0 < q < t, &\text{only }m_q\text{ and }m_{q+1}\text{ will be recovered \textit{incorrectly}}\\
q = t, &\text{only }m_t\text{ will be recovered \textit{incorrectly}}
\end{cases}
\end{equation*}

\section*{Problem 3}

We assume that we know that $m_2$ is of the form $m_2 = x || y || z || w || \texttt{0x03}^3$,
where we have determined $z$ and $w$, but do not know $x$ and $y$.
Note that if we modify $c_1$ as
\begin{align*}
\Delta_i
    &= \texttt{0x00} || i || (z \oplus \texttt{0x06}) || (w \oplus \texttt{0x06}) || (\texttt{0x03} \oplus \texttt{0x06})^3\\
c_1'
    &= c_1 \oplus \Delta_i
\end{align*}
we yield ciphertext which corresponds to
\begin{align*}
m_2'
    &= c_1' \oplus F_k^{-1}(c_2)\\
    &= \Delta_i \oplus m_2\\
    &= x || (y \oplus i) || \texttt{0x06}^5
\end{align*}
By checking against the padding oracle,
we can verify whether or not $y \oplus i = \texttt{0x06}$.
In summary,
to compute $y$, we can use the following algorithm, which returns the value of $y$:
\begin{algorithm}[H]
\begin{algorithmic}
\Procedure{SolveY}{}
    \For{$i \in \set{ 0, 1 }^8$}
        \State Set $\Delta_i \gets \texttt{0x00} || i || (z \oplus \texttt{0x06}) || (w \oplus \texttt{0x06}) || (\texttt{0x03} \oplus \texttt{0x06})^3$
        \State Compute $c_1' \gets c_1 \oplus \Delta_i$
        \State Send $\langle IV, c_1', c_2 \rangle$ to the padding oracle
        \If{The oracle reports \textbf{no} padding error}
            \State \Return $y = i \oplus \texttt{0x06}$
        \EndIf
    \EndFor
\EndProcedure
\end{algorithmic}
\end{algorithm}

\section*{Problem 4}

Let $m = \langle m_1, \ldots, m_t \rangle$ be a message,
$c = \langle c_0, c_1, \ldots, c_t \rangle = \langle IV, c_1, \ldots, c_t \rangle$ the corresponding ciphertext,
produced with the CTR mode of operation.
Assume we have the ciphertext and a padding oracle and wish to determine the plaintext.
Let $n$ be the block length and note that if we pass the modified $c_t' = c_t \oplus \Delta$,
$\Delta \in \set{ 0, 1 }^n$ to the padding oracle,
it will ``see'' the final block of the message as $(c_t \oplus \Delta) \oplus F_k(IV + t) = (c_t \oplus F_k(IV + t)) \oplus \Delta = m_t \oplus \Delta$.
With this ability, we can compute all of $m$.
In particular, \textbf{we can follow the same approach as in the slides with the change that,
\textit{instead} of modifying the penultimate ciphertext block ($c_1$ in the slides)
to ``present'' a modified version of the final plaintext block ($m_2$ in the slides) to the padding oracle,
we would now make the modifications to the final ciphertext clock ($c_2$ in the slides)}.
The details are outlined in the following sections.

\pagebreak

\subsection*{Determining padding length}

First, we will determine the padding length, $b$.
To do this, we choose the appropriate $\Delta$ to modify bytes in $c_t$ from left to right until a padding error is returned by the oracle,
in a similar to manner as we did for CBC mode.
The following algorithm returns the padding length
(the only real change from the slides is to modify the final block of ciphertext instead of the penultimate block):
\begin{algorithm}[H]
\begin{algorithmic}
\Procedure{SolvePadding}{}
    \For{$i \in \set{ 1, \ldots, n }$}
        \State Set $\Delta_i \gets \texttt{0x01}^i || \texttt{0x00}^{n - i}$
        \State Compute $c_t' \gets c_t \oplus \Delta_i$ (alter the first $i$ bytes of $c_t$)
        \State Send $\langle c_0, c_1, \ldots, c_t' \rangle$ to the padding oracle
        \If{The oracle reports a padding error}
            \State \Return $b = n - (i - 1)$
        \EndIf
    \EndFor
\EndProcedure
\end{algorithmic}
\end{algorithm}

\subsection*{Determining bytes of plaintext}

We can know determine the bytes of the plaintext,
in order,
starting from the rightmost non-padding byte.
For the sake of concreteness,
we will use the example on the slides and assume our plaintext/ciphertext pair is
$m = \langle m_1, m_2 \rangle$,
$c = \langle IV, c_1, c_2 \rangle$,
where $m_2 = x || y || z || w || \texttt{0x03}^3$
(we determined $b = 3$ in the step above).
We can try different $\Delta$ until the modified $m_2'$ ``seen'' by the padding oracle is appropriately padded for padding length 4,
in a similar manner as we did for CBC mode.
The following algorithm returns $w$
(the only real change from the slides is to modify $c_2$ instead of $c_1$)\footnote{the algorithm is then easily repeated to return $z$, and then $y$, etc.}:
\begin{algorithm}[H]
\begin{algorithmic}
\Procedure{SolveW}{}
    \For{$i \in \set{ 0, 1 }^8$}
        \State Set $\Delta_i \gets \texttt{0x00}^3 || i || (\texttt{0x03} \oplus \texttt{0x04})^3$
        \State Compute $c_2' \gets c_2 \oplus \Delta_i$
        \State Send $\langle IV, c_1', c_2 \rangle$ to the padding oracle
        \If{The oracle reports \textbf{no} padding error}
            \State \Return $w = i \oplus \texttt{0x04}$
        \EndIf
    \EndFor
\EndProcedure
\end{algorithmic}
\end{algorithm}

\end{document}
